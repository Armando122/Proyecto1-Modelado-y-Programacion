\documentclass[]{article}

%opening
\title{README}
\author{Armando Ramírez González - 317158225 \\ Cecilia Villatoro Ramos
- 419002938}

\begin{document}

\maketitle

\section{Instrucciones de ejecución}
Para la ejecución del proyecto es necesario contar con una llave exclusiva de
OpenWeatherMap, el programa contiene una llave por defecto en caso de no
haber llave se debe conseguir una registrándose en la página del web service.
Las bibliotecas utilizadas fueron csv, json, requests y waiter sobre el lenguaje
de programación Python en su versión $ 3.7.5 $ en caso de contar con ellas
es necesario instalarlas mediante pip3, en caso de no tener pip3 es necesario
instalarlo siguiendo las instrucciones correspondientes para cada sistema
operativo.
La forma de ejecutar el proyecto y las pruebas son:
\begin{center}
	\textbf{Ejecución:} python3 proyecto1.py \\
	Devolverá el resultado en la salida estándar.\\
\end{center}
\begin{center}
	\textbf{Pruebas:} python3 tests.py
\end{center}

\section{Definición del problema}
Fuimos contratados por el aeropuerto de la Ciudad de México, para realizar
un programa que devolviera el clima de las ciudades destino de 3 mil tickets
diferentes que salen el mismo día que se corre el algoritmo. No es necesaria
la interacción con el usuario, sólo la impresión de los siguientes datos del
clima de cada ciudad: nombre de la ciudad o el aeropuerto, la descripción
general del clima, temperatura, sensación térmica, temperaturas mínima y
máxima, presión y humedad.


\section{Análisis del problema}
Para este problema los datos con que se cuentan son los siguientes:
\begin{itemize}
	\item \textbf{Datos de entrada:}
	\begin{itemize}
		\item Dataset1.csv que contiene las ciudades o países de origen y
		destino escritos en clave IATA correspondiente a la ISO 3166, además
		contiene las latitudes y longitudes del origen y del destino.
		\item Dataset2.csv que contiene el destino, hora de salida, hora de
		llegada y la fecha de salida.
	\end{itemize}
\end{itemize}

Y los datos que queremos que nos de el programa son:
\begin{itemize}
	\item \textbf{Datos de salida:}
	\begin{itemize}
		\item Informe del clima. Que contenga: nombre de la ciudad,
		descripción del clima, temperatura mínima, temperatura máxima y
		humedad humedad de la ciudad origen.
	\end{itemize}
\end{itemize}

Se necesita una manera eficaz de leer los 3 mil datos distintos, guardarlos y
poder accesar a ellos de manera rápida y directa. De la misma manera, es
necesaria una forma de emparejar el clima con su ciudad correspondiente.

Para devolver el clima, se hizo uso de el API de OpenWeatherMap. Este API
solo permite hasta 60 peticiones por minuto, así que se necesita una
manera de dosificar las peticiones y medir el tiempo.

Como no es necesaria la interacción con el usuario, se necesita
una manera de informar los climas de todas las ciudades sin distinción del
ticket en el que aparezcan. Es decir, sin emparejar ciudad origen con
destino. Por lo tanto, una opción es recolectar todas las ciudades
diferentes que aparezcan en los tickets y emparejarlas con su clima, así
se evitan hacer dos llamadas para una misma ciudad.

Al ver los \emph{dataset} recibidos, se observan que \emph{dataset2} incluye
datos innecesarios para resolver el problema, como los horarios de salida y
llegada. Además, se nota que algunos tickes de \emph{dataset1} incluyen un
origen diferente a la Ciudad de México, iguamente se tomaron esas ciudades
en cuenta para devolver el clima. Para los vuelos nacionales las ciudades
están en código IATA, pero como el API escogida no usa esos datos, se usan
las coordenadas que el \emph{dataset1} incluye para cada ciudad.

Al hacer peticiones, hay que tomar en cuenta varios casos. Si la petición
es exitosa, se manipulan los datos recibidos para devolver sólo lo
necesario. Si la petición tiene algún problema, se le avisa al usuario para
que vuelva a correr el programa.

Otra fase del problema es la impresión del clima. Para imprimir los datos, es
necesario un formato legible por cualquier persona. De ser posible, en
español.

En conclusión, podemos observar tres etapas principales del problema:
el manejo, almacenamiento y acceso de los datos recibidos, el proceso de
realizar una petición a OpenWeatherMap y la impresión de la información
requerida que se decidió que sería en orden alfabético.

\section{Selección de la mejor alternativa}
Se escogió el lenguaje Python por su versatilidad, ya que al ser un lenguaje
multiparadigma se puede construir el proyecto con una estructura Orientada
a Objetos, también nos provee de herramientas sencillas y útiles para
trabajar con web services. Además incluye todas las esctructuras que fueron
necesarias para resolver el problema, como los diccionarios.

Como ya fue mencionado antes, el nombre de las ciudades nacionales estaba
en código IATA, lo que no hace posible hacer una petición a
OpenWeatherMap con este dato, pero el \emph{dataset1} también contaba
con las coordenadas de cada ciudad, un dato con el que sí es posible hacer
una petición. Por esta razón era necesario guardar los dos datos para cada
ciudad, para eso se usaron diccionarios, donde la clave de búsqueda es el
IATA de la ciudad y el contenido es una lista de sus coordenadas.

Los diccionarios son convenientes en este caso, porque si la ciudad se repite,
sólo se reescriben las coordenadas, pero sigue existiendo una única entrada
para cada ciudad.

Las ciudades de \emph{dataset2} también se guardaron en un diccionario,
pero con entradas vacías porque este no contenía otros datos relevantes a
parte del nombre de las ciudades.

En el código, la lectura de los archivos y la escritura en los diccionarios se
hizo con un sólo método \emph{lectura()} que distingue entre
\emph{dataset1.csv} y \emph{dataset2.csv}.

Para la etapa de las peticiones, se consideró que el API de OpenWeatherMap
es una herramienta que devuelve los datos necesarios para este problema,
además es de muy fácil acceso, gratuita y se maneja de manera sencila con
Python.

Se realizó un método llamado \emph{peticiones()} que recibe un diccionario,
en este caso recibe la unión de los dos diccionarios creados con anterioridad
a partir de los \emph{dataset}. Con un ciclo \emph{for} realiza una petición
por entrada del diccionario, si la petición suelta un error por el nombre en
código IATA, hace otra petición con las coordenadas. Se verificó que en el
caso de las ciudades del \emph{dataset2} no ocurriera ningún error por el
nombre de la ciudad. Se auxilia con un contador para no realizar más de 30
peticiones cada 30 segundos, así el programa espera 30 segundos en lugar
de un minuto y no satura de peticiones al API. Para esa espera, se utilizó la
librería \emph{waiters} de Python, ya que sirve para esta ocasión y es la que
se conocía. La información de cada petición se empareja en otro dicionario
con el nombre de la ciudad, este diccionario servirá para la impresión de la
información requerida.

Como las peticiones regresan más información de la pedida, en el método
\emph{impresión()} se filtra, y sólo son impresos los datos requeridos. Como
no es necesaria la distición entre ciudad origen y destino, simplemente se
imprime el clima de toda ciudad que aparezca en los \emph{dataset}.

\section{Diagrama de flujo o pseudocódigo}
\textbf{Proyecto1}\\
\textbf{Inicio:}\\
\indent \textbf{Leer} archivo csv.\\
\indent\indent diccionario = \{\}\\
\indent\indent \textbf{if} dataset1 existe\\
\indent\indent\indent diccionario $ \leftarrow $ \{dataset1.ciudadOrigen :
[ciudad.latitud, ciudad.longitud]\} \\
\indent\indent\indent diccionario $ \leftarrow $ \{dataset1.ciudadDestino :
[ciudad.latitud,ciudad.longitud]\}
\indent\indent \textbf{elif} dataset2 existe.\\
\indent\indent\indent diccionario $ \leftarrow $ \{dataset2.destino : []\}\\
\indent\indent \textbf{else:}\\
\indent\indent\indent print Archivo no encontrado.\\
\indent \textbf{Peticiones} diccionario\\
\indent\indent $ i $ $ = $ $ 0 $\\
\indent\indent diccionarioNuevo = \{\} \\
\indent\indent \textbf{for} llave en diccionario\\
\indent\indent\indent \textbf{if} $ i $ $ >= $ $ 30 $\\
\indent\indent\indent\indent \textbf{Esperar} 30 segundos\\
\indent\indent\indent\indent $ i $ $ = $ $ 0 $\\
\indent\indent\indent \textbf{Hacer} petición $ \leftarrow $ petición a API
con llave\\
\indent\indent\indent \textbf{if} petición $ == $ \textbf{True}\\
\indent\indent\indent\indent diccionarioNuevo $ \leftarrow $ \{llave :
peticiónClave\}\\
\indent\indent\indent\indent $ i $ $ += $ $ 1 $\\
\indent\indent\indent \textbf{elif} petición $ == $ \textbf{False}\\
\indent\indent\indent\indent diccionarioNuevo $ \leftarrow $ \{llave :
peticiónCoordenadas\}\\
\indent\indent\indent\indent $ i $ $ += $ $ 1 $\\
\indent\indent\indent \textbf{else}\\
\indent\indent\indent\indent diccionarioNuevo $ \leftarrow $ \{llave : error\}
\\
\indent\indent\indent\indent $ i $ $ += $ $ 1 $\\
\indent\textbf{Imprimir} diccionario\\
\indent\indent \textbf{for} llave in diccionario\\
\indent\indent\indent \textbf{if} diccionario[llave] $ != $ error\\
\indent\indent\indent\indent \textbf{Print} diccionario[llave] + información
de clima\\
\indent\indent\indent \textbf{else}\\
\indent\indent\indent\indent \textbf{Print} diccionario[llave] + error\\
\textbf{Fin}

\section{Pensamiento a futuro}
Algunas mejoras que se le pueden hacer al proyecto es solucionar todos los
problemas que presenta, como que el API no encontró el idioma para algunas
ciudades. Por otro lado, a los datos impresos del clima se le puede agregar
más información, como la zona horaria, el porcentaje de nubes, la visibilidad
y la velocidad del viento. Esta mejora es sencilla y se puede hacer en poco
tiempo, de acuerdo a lo que decida el cliente.

Usando varios hilos de ejecución se pueden hacer más peticiones en menos
tiempo y así el programa sería más veloz. También, al ser dos
programadores trabajando, se podría usar ambas llaves de acceso al API al
mismo tiempo.

Con una membresía diferente de OpenWeatherMap, que tiene costo, se
podría tomar en cuenta el tiempo de vuelo y pedir predicciones del clima a la
hora de llegada del vuelo a la ciudad en cuestión.

También se podría construir una interfaz de usuario que permita ingresar
el nombre de la ciudad y se imprima la información del clima uno a uno, o,
que se pueda elegir de qué ciudades se quiere ver la información, en lugar de
que se impriman todos al mismo tiempo.

Por último están las actualizaciones a nuestro programa y mantenimiento
que este pueda llegar a necesitar, ya sea por cambios en la API de
OpenWeatherMap y como ésta es utilizada o futuras incompatibilidades que
se puedan llegar a presentar en la ejecución con nuevas versiones del
lenguaje.

En relación al costo del código se propone la cantidad de $ 3500.00 $ pesos
mexicanos y un costo extra para cada actualización, después de la primera o
mantenimiento que llegue a necesitar el programa, según se determine con
el cliente.

\section{Herramientas de trabajo}
Para la realización de este proyecto se utilizaron las bibliotecas de Python: unittest (para la ejecución de las pruebas), csv (para la lectura de los archivos csv), json (para procesar las solicitudes al web service), requests (para hacer solicitudes al web service y recibir la información) y waiter (para limitar el número de peticiones al web servie y no saturarlo) cuya documentación necesaria para utilizarlas se encuentra en $ https://docs.python.org/es/3.7/ $.

Para hacer las peticiones del clima se utilizó el web service de OpenWeatherMap cuya información de uso para obtener el clima del momento en que se corre el algoritmo se puede consultar en: $ https://openweathermap.org/current $, dentro de la documentación del web service se encontró que una forma de escribir el nombre del país es mediante la clave del país dada por la ISO 3166 o clave IATA (International Air Transport Association) propuesta por el organismo del mismo nombre, pero está clave no es válida para ciudades por lo que se buscó un código o base de datos que pudiese convertir el nombre de cada ciudad en clave IATA a su nombre común, tal código que se encontró y se utilizó fue el del repositorio de github de $ ckimrie $ cuya dirección es: $ https://gist.github.com/ckimrie/4755385 $ los datos ahí se procesaron y se colocaron en un archivo csv para poder utilizarlo con el programa.

Finalmente se utilizaron dos datasets con la información necesaria sobre los vuelos para poder ejecutar el programa estos datasets fueron brindados por la ayudante de la materia de Modelado y Programación.

\end{document}
