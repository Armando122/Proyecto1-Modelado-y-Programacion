\documentclass[]{article}

%opening
\title{README}
\author{Armando Ramírez González - 317158225 \\ Cecilia Villatoro Ramos 
- 419002938}

\begin{document}

\maketitle

\section{Definición del problema}

Fuimos contratados por el aeropuerto de la Ciudad de México, para realizar 
un programa que devolviera el clima de las ciudades destino de 3 mil tickets 
diferentes, fueron entregados en dos arcivos \emph{.csv}, uno de vuelos 
nacionales, llamado \emph{dataset1} y uno de vuelos internacionales llamado 
\emph{dataset2}. 

No es necesaria la interacción con el usuario, sólo la impresión de los 
sigiuentes datos del clima de cada ciudad: nombre de la ciudad o el 
aeropuerto, la descripción general del clima, tempertatura, sensación términa, 
temperaturas mínima y máxima, presión y humedad.

\section{Análisis del problema}

Se necesita una maera eficaz de leer los 3 mil datos distintos, guardarlos y 
poder accedar a ellos de manera rápida y directa. De la misma manera, es 
necesaria una forma de emparejar el clima con su ciudad correspondiente.

Para devolver el clima, se hizo uso de el API de OpenWeatherMap. Este API
solo permite hasta 60 peticiones por minuto, así que se necesita una 
manera de dosificar las peticiones y medir el tiempo.

Como no es necesaria la interacción con el usuario, se necesita 
una manera de informar los climas de todas las ciudades sin distinción del 
ticket en el que aparezcan. Es decir, sin emparejar ciudad origen con 
destino. Por lo tanto, una opción es recolectar todas las ciudades 
diferentes que aparezcan en los tickets y emparejarlas con su clima, así 
se evitan hacer dos llamadas para una misma ciudad.

Al ver los \emph{dataset} recibidos, se observan que \emph{dataset2} incluye 
datos innecesarios para resolver el problema, como los horarios de salida y 
llegada. Además, se nota que algunos tickes de \emph{dataset1} incluyen un 
origen diferente a la Ciudad de México, iguamente se tomaron esas ciudades 
en cuenta para devolver el clima. Para los vuelos nacionales las ciudades están 
en código IATA, pero como el API escogida no usa esos datos, se usan las 
coordenadas que el \emph{dataset} incluye para cada ciudad. 

Al hacer peticiones, hay que tomar en cuenta varios casos. Si la petición 
es exitosa, se manipulan los datos recibidos para devolver sólo lo 
necesario. Si la petición tiene algún problema, se le avisa al usuario para 
que vuelva a correr el programa. 

Otra fase del problema es la impresión del clima. Para imprimir los datos, es 
necesario un formato legible por cualquier persona. De ser posible, en 
español.

En conclusión, podemos observar tres etapas principales del problema: 
el manejo, almacenamiento y acceso de los datos recibidos, el proceso de 
realizar una petición a OpenWeatherMap y la impresión de la información 
requerida.

\section{Selección de la mejor alternativa}

Se escogió el lenguaje Python por su versatilidad, ya que al ser un lenguaje 
multiparadigma se puede construir el proyecto con una estructura Orientada 
a Objetos, también nos provee de herramientas sencillas y útiles para trabajar 
con web services. Además incluye todas las esctructuras que fueron 
necesarias para resolver el problema, como los diccionarios.

Como ya fue mencionado antes, el nombre de las ciudades nacionales estaba 
en código IATA, lo que no hace posible hacer una petición a OpenWeatherMap 
con este dato, pero el \emph{dataset1} también contaba con las coordenadas 
de cada ciudad, un dato con el que sí es posible hacer una petición. Por esta 
razón era necesario guardar los dos datos para cada ciudad, para eso se 
usaron diccionarios, donde la clave de búsqueda es el IATA de la ciudad y el 
contenido es una lista de sus coordenadas.

Los diccionarios son convenientes en este caso, porque si la ciudad se repite, 
sólo se reescriben las coordenadas, pero sigue existiendo una única entrada 
para cada ciudad.

Las ciudades de \emph{dataset2} también se guardaron en un diccionario, 
pero con entradas vacías porque este no contenía otros datos relevantes a 
parte del nombre de las ciudades.

En el código, la lectura de los archivos y la escritura en los diccionarios se 
hizo con un sólo método \emph{lectura()} que distingue entre
\emph{dataset1.csv} y \emph{dataset2.csv}.

Para la etapa de las peticiones, se consideró que el API de OpenWeatherMap 
es una herramienta que devuelve los datos necesarios para este problema, 
además es de muy fácil acceso, gratuita y se maneja de manera sencila con 
Python.

Se realizó un método llamado \emph{peticiones()} que recibe un diccionario, 
en este caso recibe la unión de los dos diccionarios creados con anterioridad 
a partir de los \emph{dataset}. Con un ciclo \emph{for} realiza una petición 
por entrada del diccionario, si la petición suelta un error por el nombre en 
código IATA, hace otra petición con las coordenadas. Se verificó que en el 
caso de las ciudades del  \emph{dataset2} no ocurriera ningún error por el 
nombre de la ciudad. Se auxilia con un contador para no realizar más de 30 
peticiones cada 30 segundos, así el programa espera 30 segundos en lugar 
de un minuto y no satura de peticiones al API. Para esa espera, se utilizó la 
librería \emph{waiters} de Python, ya que sirve para esta ocasión y es la que 
se conocía. La información de cada petición se empareja en otro dicionario 
con el nombre de la ciudad, este diccionario servirá para la impresión de la 
información requerida.

Como las peticiones regresan más información de la pedida, en el método
\emph{impresión()} se filtra, y sólo son impresos los datos requeridos. 

\section{Diagrama de flujo o pseudocódigo}
\section{Pensamiento a futuro}

\end{document}
