\documentclass[]{article}

%opening
\title{README}
\author{Armando Ramírez González - 317158225 \\ Cecilia Villatoro Ramos - 
419002938}

\begin{document}

\maketitle

\section{Definición del problema}

Fuimos contratados por el aeropuerto de la Ciudad de México, para realizar un programa que devolviera el clima de las ciudades destino de 3 mil tickets diferentes. No es necesaria la interacción con el usuario, sólo la impresión de los climas.

\section{Análisis del problema}

Se necesita una maera eficaz de leer los 3 mil datos distintos, guardarlos y 
poder accedar a ellos de manera rápida y directa. De la misma manera, es 
necesaria una forma de emparejar el clima con su ciudad correspondiente.

Para devolver el clima, se hizo uso de el API de OpenWeatherMap. Este API
solo permite hasta 60 peticiones por minuto, así que se necesita una 
manera de dosificar las peticiones y medir el tiempo.

Como no es necesaria la interacción con el usuario, se necesita 
una manera de informar los climas de todas las ciudades sin distinción del 
ticket en el que aparezcan. Es decir, sin emparejar ciudad origen con 
destino. Por lo tanto, una opción es recolectar todas las ciudades 
diferentes que aparezcan en los tickets y emparejarlas con su clima, así 
se evitan hacer dos llamadas para una misma ciudad.

Al ver los \emph{dataset} recibidos, se nota que algunos tickes incluyen 
un origen diferente a la Ciudad de México, iguamente se tomaron esas 
ciudades en cuenta para devolver el clima. Para los vuelos nacionales las 
ciudades están en código IATA, pero como el API escogida no usa esos 
datos, se usan las coordenadas que el \emph{dataset} incluye para cada 
ciudad. 

Al hacer peticiones, hay que tomar en cuenta varios casos. Si la petición 
es exitosa, se manipulan los datos recibidos para devolver sólo lo 
necesario. Si la petición tiene algún problema, se le avisa al usuario para 
que vuelva a correr el programa. 

Otra fase del problema es la impresión del clima. Para imprimir los datos, es 
necesario un formato legible por cualquier persona. De ser posible, en 
español.

En conclusión, podemos observar tres etapas principales del problema: 
el manejo, almacenamiento y acceso de los datos recibidos, el proceso de 
realizar una petición a OpenWeatherMap y la impresión de la información 
requerida.

\section{Selección de la mejor alternativa}
\section{Diagrama de flujo o pseudocódigo}
\section{Pensamiento a futuro}

\end{document}
